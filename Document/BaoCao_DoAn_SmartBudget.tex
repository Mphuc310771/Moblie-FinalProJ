\documentclass[12pt,a4paper]{article}
\usepackage[utf8]{inputenc}
\usepackage[T5]{fontenc}
\usepackage[vietnamese]{babel}
\usepackage{vntex}
\usepackage{geometry}
\usepackage{graphicx}
\usepackage{float}
\usepackage{hyperref}
\usepackage{xcolor}
\usepackage{booktabs}
\usepackage{enumitem}
\usepackage{fancyhdr}
\usepackage{titlesec}
\usepackage{tikz}
\usepackage{multicol}
\usetikzlibrary{shapes.geometric, arrows.meta, positioning}

\geometry{left=3cm, right=2cm, top=2cm, bottom=2cm}
\definecolor{smartgreen}{RGB}{40, 167, 69}
\definecolor{primaryblue}{RGB}{0, 123, 255}
\definecolor{darkblue}{RGB}{0, 82, 155}

\pagestyle{fancy}
\fancyhf{}
\fancyhead[L]{\small\leftmark}
\fancyhead[R]{\small\textbf{SmartBudget}}
\fancyfoot[C]{\thepage}

\titleformat{\section}{\normalfont\LARGE\bfseries\color{darkblue}}{\thesection}{1em}{}
\titleformat{\subsection}{\normalfont\Large\bfseries\color{smartgreen}}{\thesubsection}{1em}{}
\titleformat{\subsubsection}{\normalfont\large\bfseries}{\thesubsubsection}{1em}{}

\hypersetup{colorlinks=true, linkcolor=smartgreen, urlcolor=primaryblue}

\begin{document}

% TRANG BÌA
\begin{titlepage}
\centering
\vspace*{1cm}
{\large \textbf{TRƯỜNG ĐẠI HỌC KHOA HỌC TỰ NHIÊN}}\\[0.1cm]
{\large \textbf{ĐẠI HỌC QUỐC GIA THÀNH PHỐ HỒ CHÍ MINH}}\\[0.3cm]
{\large KHOA CÔNG NGHỆ THÔNG TIN}\\[2cm]
\includegraphics[width=5cm]{logo_hcmus.png}\\[1.5cm]
\rule{\linewidth}{0.5mm}\\[0.4cm]
{\Huge \textbf{BÁO CÁO ĐỒ ÁN CUỐI KỲ}}\\[0.2cm]
{\Large \textbf{Phát Triển Ứng Dụng Di Động}}\\[0.4cm]
\rule{\linewidth}{0.5mm}\\[1.5cm]

{\fontsize{30}{36}\selectfont\bfseries\color{smartgreen} SmartBudget}\\[0.5cm]
{\Large Ứng dụng Quản lý Chi tiêu Cá nhân Thông minh}\\[0.2cm]
{\large với tích hợp Trí tuệ Nhân tạo (AI)}\\[2cm]

\begin{minipage}{0.48\textwidth}
\begin{flushleft}
\textbf{\large SINH VIÊN THỰC HIỆN:}\\[0.3cm]
Đỗ Trần Minh Phúc - 23120156\\
Hoàng Kim Trí - 23120098\\[0.2cm]
Lớp: 23CNTN
\end{flushleft}
\end{minipage}
\hfill
\begin{minipage}{0.45\textwidth}
\begin{flushright}
\textbf{\large GIẢNG VIÊN HƯỚNG DẪN:}\\[0.3cm]
Trần Minh Triết\\
Đỗ Trọng Lễ\\
Huỳnh Viết Thám
\end{flushright}
\end{minipage}

\vfill
{\large TP. Hồ Chí Minh, Tháng 01/2026}
\end{titlepage}

\tableofcontents
\newpage

%==============================================================
\section{Giới thiệu tổng quan}
%==============================================================

\subsection{Đặt vấn đề}

Trong bối cảnh xã hội hiện đại, việc quản lý tài chính cá nhân ngày càng trở nên quan trọng. Nhiều người, đặc biệt là sinh viên và người mới đi làm, gặp khó khăn trong việc theo dõi các khoản thu chi hàng ngày. Điều này dẫn đến tình trạng chi tiêu vượt quá khả năng tài chính, không thể tiết kiệm được, hoặc không biết tiền đã đi đâu.

Các phương pháp truyền thống như ghi chép sổ tay hoặc Excel tỏ ra bất tiện và tốn thời gian. Người dùng cần một giải pháp hiện đại - ứng dụng di động có thể sử dụng mọi lúc mọi nơi, tự động hóa việc nhập liệu, và cung cấp các phân tích thông minh giúp hiểu rõ hơn về thói quen chi tiêu của mình.

Bên cạnh đó, sự phát triển của công nghệ Trí tuệ Nhân tạo (AI) mở ra cơ hội để xây dựng các tính năng tư vấn tài chính cá nhân hóa, tự động trích xuất thông tin từ hóa đơn, giúp tiết kiệm thời gian và nâng cao trải nghiệm người dùng.

\subsection{Mục tiêu dự án}

Dự án SmartBudget được phát triển nhằm đạt được các mục tiêu sau:

\begin{enumerate}[leftmargin=*]
\item \textbf{Xây dựng ứng dụng quản lý chi tiêu toàn diện:} Cho phép người dùng ghi chép, phân loại và theo dõi các khoản thu chi một cách dễ dàng thông qua giao diện trực quan, thân thiện.

\item \textbf{Tích hợp trí tuệ nhân tạo:} Sử dụng các mô hình AI tiên tiến (Google Gemini, Groq Llama) để cung cấp tính năng tư vấn tài chính, phân tích thói quen chi tiêu, và tự động trích xuất thông tin từ hóa đơn bằng công nghệ OCR.

\item \textbf{Trực quan hóa dữ liệu:} Hiển thị báo cáo thống kê bằng biểu đồ Pie Chart và Bar Chart giúp người dùng nhanh chóng nắm bắt tình hình tài chính.

\item \textbf{Hỗ trợ lập kế hoạch tài chính:} Cung cấp tính năng thiết lập ngân sách và mục tiêu tiết kiệm, theo dõi tiến độ và cảnh báo khi gần vượt ngân sách.

\item \textbf{Đồng bộ đa thiết bị:} Tích hợp Firebase để người dùng có thể truy cập dữ liệu từ nhiều thiết bị và không lo mất dữ liệu.

\item \textbf{Thiết kế giao diện hiện đại:} Áp dụng Material Design 3 kết hợp phong cách Neon Glassmorphism độc đáo với hỗ trợ Dark Mode hoàn chỉnh.
\end{enumerate}

\subsection{Đối tượng sử dụng}

\begin{itemize}[leftmargin=*]
\item \textbf{Sinh viên:} Quản lý chi tiêu hàng tháng với nguồn tài chính hạn hẹp từ trợ cấp gia đình hoặc làm thêm, đặt mục tiêu tiết kiệm cho các nhu cầu như du lịch, mua sắm thiết bị.

\item \textbf{Người đi làm:} Theo dõi các khoản chi cố định (nhà ở, điện nước, xăng xe) và biến động, phân tích xu hướng chi tiêu để điều chỉnh hành vi tài chính.

\item \textbf{Hộ gia đình:} Quản lý ngân sách gia đình, theo dõi chi tiêu theo các danh mục như thực phẩm, giáo dục, y tế, giải trí.

\item \textbf{Freelancer:} Theo dõi thu nhập từ nhiều nguồn khác nhau, quản lý dòng tiền không ổn định.
\end{itemize}

\subsection{Phạm vi và tính năng}

Ứng dụng SmartBudget bao gồm 11 tính năng chính được chia thành các nhóm:

\subsubsection{Nhóm tính năng cốt lõi}

\begin{enumerate}[leftmargin=*]
\item \textbf{Ghi chép thu chi:} Cho phép thêm, sửa, xóa các giao dịch. Mỗi giao dịch bao gồm: số tiền, danh mục (14 danh mục mặc định với emoji icon và màu sắc riêng), ngày tháng (DatePicker), ghi chú tùy chọn, và có thể đính kèm ảnh hóa đơn. Hỗ trợ phân biệt rõ ràng giữa chi tiêu (9 danh mục: Ăn uống, Di chuyển, Mua sắm, Sức khỏe, Giải trí, Học tập, Nhà cửa, Điện nước, Khác) và thu nhập (5 danh mục: Lương, Quà tặng, Đầu tư, Thưởng, Khác).

\item \textbf{Dashboard tổng quan:} Màn hình chính được thiết kế với phong cách hiện đại. Điểm nhấn là \textbf{Thẻ số dư (Total Balance Card)} được thiết kế dạng thẻ tín dụng cao cấp với hiệu ứng sóng nước 3D (Fluid Waves). Danh sách 5 giao dịch gần nhất được hiển thị trực quan. Ba nút hành động nhanh dẫn đến các tính năng AI và OCR.

\item \textbf{Báo cáo thống kê chi tiết:} Cung cấp bốn tùy chọn khoảng thời gian chi tiết. Sử dụng thư viện MPAndroidChart với khả năng tùy biến màu sắc cao.
\end{enumerate}

\subsubsection{Nhóm tính năng lập kế hoạch}

\begin{enumerate}[leftmargin=*]
\setcounter{enumi}{3}
\item \textbf{Quản lý ngân sách:} Cho phép thiết lập hạn mức chi tiêu. Hệ thống cảnh báo trực quan với màu sắc (Xanh/Vàng/Đỏ) tùy theo mức độ sử dụng ngân sách.

\item \textbf{Mục tiêu tiết kiệm:} Tạo và theo dõi các mục tiêu tiết kiệm dài hạn (Mua nhà, Du lịch...). Hiển thị tiến độ % completion và animation chúc mừng khi hoàn thành.

\item \textbf{Chi tiêu định kỳ:} Quản lý các khoản chi lặp lại hàng tháng (Netflix, Spotify, Tiền nhà). Tự động nhắc nhở và tính toán ngày đến hạn tiếp theo.
\end{enumerate}

\subsubsection{Nhóm tính năng AI}

\begin{enumerate}[leftmargin=*]
\setcounter{enumi}{6}
\item \textbf{Trợ lý AI tư vấn tài chính:} Chatbot thông minh tích hợp Google Gemini và Groq. AI có khả năng hiểu ngữ cảnh tài chính của người dùng để đưa ra lời khuyên cá nhân hóa (Personalized Advice) thay vì các câu trả lời chung chung.

\item \textbf{Quét hóa đơn OCR thông minh:} Kết hợp sức mạnh của ML Kit (nhận diện văn bản) và AI (phân tích ngữ nghĩa). Tự động trích xuất chính xác 95\% thông tin từ hóa đơn siêu thị, quán cafe gồm: Số tiền, Ngày tháng, và Tên cửa hàng. Cơ chế fallback thông minh giúp hệ thống vẫn hoạt động ngay cả khi offline (sử dụng Regex).
\end{enumerate}

\subsubsection{Nhóm tính năng hệ thống}

\begin{enumerate}[leftmargin=*]
\setcounter{enumi}{9}
\item \textbf{Đồng bộ Firebase Cloud:} Cơ chế Offline-First: cho phép sử dụng app khi mất mạng, dữ liệu sẽ tự động đồng bộ lên Firestore ngay khi có Internet. Đảm bảo toàn vẹn dữ liệu.

\item \textbf{Nhắc nhở thông minh:} Cài đặt lịch nhắc ghi chép chi tiêu để tạo thói quen tốt.

\item \textbf{Hệ thống giao diện thích ứng (Adaptive Theming):} Ứng dụng không chỉ đổi màu nền đơn thuần mà có khả năng thay đổi toàn bộ assets hình ảnh theo chế độ sáng tối. Ví dụ: Thẻ ngân hàng chuyển từ màu "Titanium Silver" sang "Cyberpunk Cyan" khi bật Dark Mode.
\end{enumerate}

\subsection{Giao diện ứng dụng}

Ứng dụng SmartBudget được thiết kế theo nguyên tắc Material Design 3 kết hợp Glassmorphism.

\subsubsection{Màn hình Dashboard (Trang chủ)}

\begin{itemize}
\item \textbf{Header:} Lời chào động và Avatar người dùng.
\item \textbf{Premium Balance Card:} Thẻ hiển thị số dư với thiết kế 3D, thay đổi theo chủ đề.
\item \textbf{AI Insight Pill:} Một "viên thuốc" thông báo nhỏ hiển thị lời khuyên nhanh từ AI (vd: "Bạn đã tiết kiệm được 10\% so với tháng trước").
\item \textbf{Floating Navigation Bar:} Thanh điều hướng nổi hiện đại.
\end{itemize}

\subsubsection{Hỗ trợ Dark Mode (Adaptive)}

Hệ thống Dark Mode thông minh (Adaptive) mang lại trải nghiệm thị giác tối ưu:
\begin{itemize}
\item \textbf{Light Mode:} Tone màu sáng, sạch (Clean & Airy), sử dụng màu Titanium và Trắng.
\item \textbf{Dark Mode:} Tone màu tối, huyền bí (Neon & Deep Navy), sử dụng màu Cyan và Neon Blue làm điểm nhấn.
\item \textbf{Dynamic Assets:} Các hình ảnh nền, icon tự động thay đổi phiên bản tương ứng để đảm bảo độ tương phản và thẩm mỹ cao nhất.
\end{itemize}

%==============================================================
\section{Công nghệ sử dụng}
%==============================================================

\subsection{Nền tảng phát triển}

\begin{table}[H]
\centering
\begin{tabular}{|l|l|p{6cm}|}
\hline
\textbf{Thành phần} & \textbf{Công nghệ} & \textbf{Chi tiết} \\
\hline
Ngôn ngữ & Java 8 & Sử dụng Desugaring Library 2.0.4 để hỗ trợ Java 8+ APIs \\
\hline
Platform & Android SDK & compileSdk 34 (Android 14), minSdk 24 (Android 7.0) \\
\hline
Kiến trúc & MVVM & Model-View-ViewModel + Repository Pattern \\
\hline
\end{tabular}
\caption{Môi trường phát triển}
\end{table}

\subsection{Kiến trúc ứng dụng}

Dự án áp dụng mô hình \textbf{MVVM (Model-View-ViewModel)} kết hợp \textbf{Repository Pattern}, được Google khuyến nghị cho ứng dụng Android hiện đại.

\textbf{Ưu điểm:} Tách biệt logic UI và business logic, dễ dàng kiểm thử và bảo trì. ViewModel giúp giữ trạng thái UI qua các thay đổi xoay màn hình.

\subsection{Thư viện và Dependencies}

\begin{table}[H]
\centering
\begin{tabular}{|l|l|p{5.5cm}|}
\hline
\textbf{Thư viện} & \textbf{Phiên bản} & \textbf{Mục đích} \\
\hline
Room & 2.6.1 & Database Local \\
\hline
Firebase BOM & 32.7.0 & Cloud Sync & Auth \\
\hline
Gemini SDK & 0.1.2 & AI Integration \\
\hline
ML Kit & 16.0.0 & OCR \\
\hline
MPAndroidChart & 3.1.0 & Biểu đồ \\
\hline
Lottie & 6.3.0 & Hiệu ứng hoạt hình \\
\hline
Konfetti & 2.0.4 & Hiệu ứng pháo giấy \\
\hline
\end{tabular}
\caption{Các thư viện chính}
\end{table}

%==============================================================
\section{Kết luận}
%==============================================================

\subsection{Kết quả đạt được}

\begin{itemize}[leftmargin=*]
\item \textbf{Hoàn thiện sản phẩm:} Ứng dụng chạy ổn định, đầy đủ 11 tính năng như thiết kế.
\item \textbf{Nâng cấp UI/UX:} Giao diện Neon Glass độc đáo, khác biệt hoàn toàn với các ứng dụng quản lý tài chính khô cứng thông thường.
\item \textbf{Công nghệ tiên tiến:} Tích hợp thành công các công nghệ mới nhất như Generative AI và OCR vào bài toán thực tế.
\end{itemize}

\subsection{Hướng phát triển}

\begin{itemize}
\item Mở rộng hỗ trợ đa tiền tệ (Multi-currency).
\item Phát triển phiên bản iOS (sử dụng Kotlin Multiplatform).
\item Tích hợp Open Banking API để liên kết tài khoản ngân hàng thực tế.
\end{itemize}

\end{document}
